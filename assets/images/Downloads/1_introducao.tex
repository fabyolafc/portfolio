\label{\part{title}}\chapter{Introdução}
%\addcontentsline{toc}{chapter}{\Uppercase{Introdução}}
\label{introducao}

De acordo com \citeonline{gomes2008aprendizagem} o ensino das linguagens de programação visa capacitar os alunos a desenvolver um conjunto de habilidades necessárias para criar programas e sistemas computacionais que possam solucionar problemas reais. No entanto, a prática tem mostrado que muitos alunos enfrentam dificuldades significativas na compreensão e aplicação de conceitos abstratos de programação, especialmente em disciplinas introdutórias. Entre as principais dificuldades estão a compreensão e a aplicação de conceitos fundamentais, como estruturas de controle, na elaboração de algoritmos para resolver problemas concretos. Diversos estudos, como os de \citeonline{tobar2001arquitetura} e \citeonline{jenkins2002difficulty}, identificam várias causas para esses desafios.

A alta taxa de evasão e a dificuldade em manter alunos em cursos de computação são problemas comuns no Brasil \citeonline{de2018dispositivos}. Pesquisadores têm demonstrado por meio da Inteligência Artificial e da Mineração de Dados que o baixo desempenho em disciplinas iniciais, especialmente em Introdução à Programação e Algoritmos, é o principal fator que contribui para essa evasão \citeonline{gonccalves2018tecnicas} e \citeonline{guerra2018case}.



%\section{Tema}

\section{Problema}

De acordo com \citeonline{bezerra2022bora} os alunos iniciantes tendem a encontrar maiores dificuldades com a programação. Conceitos fundamentais como matemática, lógica e habilidades de resolução de problemas são frequentemente negligenciados no campo da educação básica. Consequentemente, alunos ingressam em cursos de Tecnologia da Informação (TIC) com conhecimento básico ou inexistente. Além dos pontos acima mencionados, a programação representa uma disciplina extremamente prática. Conforme indicado por \citeonline{paiva2020fostering}, estudantes inexperientes geralmente se sentem sobrecarregados com a infinidade de habilidades necessárias para lidar com problemas computacionais, habilidades que só são aprimoradas por meio de  prática. Logo, quando confrontados com vários obstáculos e contratempos iniciais, os alunos tendem a perder a motivação para se envolver com o assunto. Uma solução potencial para esse problema, conforme proposto por \citeonline{paiva2020fostering}, envolve o aumento da carga horária nas disciplinas de programação.

%\section{Hipótese (opcional)}

\section{Objetivos}

    \subsection{Objetivo geral}
    Diante desse cenário, este estudo tem como objetivo desenvolver uma plataforma de ensino de programação para iniciantes. A plataforma tem como proposta apresentar os conceitos fundamentais de programação, fornecer atividades práticas para aplicação desses conceitos e disponibilizar materiais complementares para aprofundamento do conhecimento. Os objetivos específicos incluem:
    
    \subsection{Objetivos específicos}
    
        %Exemplo de como colocar tópicos em formato de lista de itens não-numerados
        \begin{itemize}
            \item identificar os conceitos fundamentais de programação que devem ser abordados na plataforma;
            \item Produzir exercícios que permitam aos alunos consolidar o que aprenderam;
            \item Construir backend e frontend da plataforma;
            \item Avaliar a eficiência da plataforma em termos de aprendizado.
        \end{itemize}


\section{Justificativa}
A escolha da temática de pesquisa se justifica pela complexidade do aprendizado de programação, que representa uma tarefa difícil para iniciantes. A dificuldade em encontrar materiais de ensino adequados pode ser um obstáculo significativo. Muitos alunos enfrentam dificuldades para encontrar recursos de qualidade que sejam claros, acessíveis e adequados ao seu nível de habilidade. Isso pode levar à frustração, desmotivação e até mesmo ao abandono do aprendizado.

Ao desenvolver uma plataforma de ensino de programação que aborda os conceitos fundamentais, oferece atividades práticas e disponibiliza materiais complementares, é possível contribuir para a melhoria da experiência de aprendizagem dos iniciantes em programação. A plataforma proposta visa auxiliar no aumento da motivação dos alunos e na facilitação da compreensão dos conceitos, de forma a reduzir as taxas de evasão e reprovação nos cursos de Tecnologia.

